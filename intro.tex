\chapter{Introduction}
\label{ch:intro}
This chapter provides an overview of steganography in general and image steganography in particular. The use of digital image steganography as a covert communication medium is also discussed. An explanation of terminology used throughout the document are listed at the end of this chapter. 
\section{Background}
The use of steganographic techniques to hide messages in everyday objects is thought to have existed for thousands of years. The Greek historian Herodotus tells the story of Demeratus who wanted to inform his friends in Greece of an impending Persian invasion  \cite{kahn1996history}. Demeratus is said to have concealed the message in writing tablets in such a way that, to the casual observer, appeared to be blank tablets covered with wax. The hidden message inscribed on the wooden tablet itself and was recovered by the recipients after melting the wax covering it. 
\par Steganographic techniques were also claimed to have been used by PoWs in World War II to communicate with friends and family in other countries. All communication in PoW camps was monitored and subject to strict censorship. Prisoners would instead write letters containing secret messages written using heat or moisture sensitive ink that could only be retrieved by applying a certain amount of heat or moisture to the letter. The letters would often contain innocuous text in regular ink to ensure that they did not arouse any suspicion when inspected.
\par A recent example of steganography is the use of invisible colour coded dots by popular printers \cite{stegprinter} to uniquely identify each page printed by a particular printer. The dots are cleverly designed to encode the printer serial number and date and time stamp and can be read using a specialised reader. According to printer manufacturers, this would allow each page to be tied to a particular printer and ostensibly provide a way to aid law enforcement in gathering evidence. 
\par Contemporary methods of digital steganography are analogous to physical steganography methods in that they use the properties of the transport medium to hide data. They rely on subtle modifications to bits that constitute digital files to hide messages in plain sight  \cite{hinson2009introduction}.  Such modifications are required to be subtle so that they do not affect the medium in a way that would cause them to be noticeable.  These  modifications can be applied to all digital mediums like video \cite{crawford2010supraliminal} and audio. If the message to be hidden is relatively short in comparison to the size of the file, this encoding ensures that the original and modified files appear exactly the same to the casual observer.
%Steg image
\begin{figure}[h!]
\includegraphics[scale=0.10]{hb1}
\includegraphics[scale=0.10]{hb1}
\caption{\emph{The images above show that there are no obvious differences between an unaltered image and the same image with data embedded in it using open source steganographic software called Stepic \cite{stepnic}. The image on the right contains the following text embedded using steganography : ``Thoughts, words, and deeds, the statute blames with reason;  But surely dreams were ne'er indicted treason \cite{burns}"}}
\label{fig:stegexample}
\end{figure} 
%Steg image
Image steganography is perhaps the most popular used form of digital steganography as a large number of freely available tools that allow one to embed data in images are easily available. These tools, described in more detail in Chapter \ref{ch:litsurvey} allow for embedding data in almost any image format. Some of these tools also provide encryption using a password derived key to provide an additional layer of security to the embedded data. The hidden message can then be recovered by the recipient with a suitable decoding tool and the correct password or decryption key. This makes image steganography a viable medium for covert communication.  

Colour images when represented as a bitmap can use anywhere between 4 to 8 bits of data to represent one colour value of a pixel (either red, green or blue). Changing the one bit for each colour value would still ensure that the image resembled the original \emph{See: Figure \ref{fig:stegexample}}. JPEG images differ from bitmap images as they represent colours using a luminance channel (to denote the intensity of colours) and two chrominance channels that represent colours. It is still possible to modify certain areas of the JPEG image without causing the image to differ from the original.  

Another interesting use of image steganography is for the purposes of copyright enforcement  \cite{kundur2002digital}. By using digital watermarks that contain copyright information,  it is possible to uniquely identify an image even after it has  been modified. These ``marks'' can be read by specialised software which can determine the original source of the image.
\section{Motivation}
\label{sec:motivation}
It would be trivial to prove the \emph{existence} of steganography if one were provided the unaltered host image along with the image carrying the steganographic payload. By calculating the binary difference of the steganographic image from the unaltered image, one could easily discover the payload embedded in the image and extract the data if the algorithm was known. This is an unrealistic expectation, however,  as the steganographic host medium will be less likely to be found along with the altered copy. Hence it is necessary to use other means for the steganalysis of images. Particularly this project explores methods that do not require the original host medium for steganalysis.
\section{Problem Statement}
\label{sec:probstatement}
\par The use of image steganography as a covert communication channel on the internet has always been a contentious topic and a source for much speculation. Analysis of a terrorist user manual, the Technical Mujahid \cite{alfajr}, by the United States Central Intelligence Agency (CIA) mentioned chapters in that outline various concealment techniques utilising image steganography. The manual describes the use of steganography as an additional layer of security for encrypted data. It is, however, difficult to ascertain whether these techniques are in use in the wild. An argument for the limited use of steganography would be that steganography is an approach that relies on security through obscurity. Security through obscurity implies the use of a secret, but insecure, method of protecting a resource. An example of this would be hiding one's room keys under the doormat in the hope that a thief would never find them there. 
\par In 2000, Provos and Honeyman \cite{provos2001detecting}  carried out an analysis of images available on eBay for steganographic content.  They argue that either there isn't  significant use of steganography on the internet or that the steganographic techniques used are undetectable by the methods used by them. They also concede that it might be possible that encryption might have been used on the data embedded in the images and that they were unable to decrypt it. Their analysis is based on a fairly large (2 million) image dataset acquired from eBay.   More recent news reports  \cite{spies2010} suggest that members of a Russian spy ring arrested in the United States in 2010 used steganography to communicate with each other. Reports speculate that these individuals posted images with steganographic content on public image hosting sites with sensitive data embedded in them. The news reports, however, did not provide any technical details regarding the steganography methods used. This project attempts to revisit the work conducted by Provos \emph{et al} using a similar setup to try and estimate the real world usage of image steganography.

\section{Terminology}
The following terms are used throughout the document to refer to various aspects of steganography. 
\begin{itemize}
\item{\emph{Steganogram:}} A portmanteau of steganography and telegram, this is used to refer to an image containing steganographic content. The content could be text, images or audio. In this report, a steganogram is used to refer to an image containing hidden text.
\item{\emph{Steganalysis:}} The process of testing a digital artifact for embedded steganographic data. This analysis may be carried out either using statistical measures or by matching the contents of the digital artifacts against a database of known steganographic signatures.
\item{\emph{Cover medium:}} Any digital medium used as a host for hiding data. Throughout this document, the cover medium is used to refer to digital imagery.
\item{\emph{JPEG image:}} Throughout the document, JPEG images are used to refer to images that use the JPEG File Interchange Format (JFIF).
\end{itemize}
 All references made to a particular component of code or an executable script use a fixed width font, similar to \texttt{this}.

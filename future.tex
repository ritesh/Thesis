\chapter{Future Work}
\label{ch:futurework}
The analysis of images for steganographic content is a vast 
area of research. This project covers static steganalysis of
a random sample of images using freely available tools. The evaluation
of the experimental setup failed to conclusively prove the existence of
steganography in the set of downloaded images. Other research that may
provide more interesting results using the same data set are outlined
below.
\section{Steganalysis Using SVMs}
\label{sec:futuresvm}
A number of studies on the detection of steganographic content have been
covered in Chapter \ref{ch:litsurvey}. Amongst these studies, the work
of Farid \cite{farid2002detecting} provides the most promising results.
Farid proposes a universal steganalysis experimental setup that utilise
non linear Support Vector Machines (SVMs) for the classification of
images into stego/non-stego categories. Classification relies on
SVMs trained with features extracted from non stego images. The classification is claimed to
be very accurate with a low false positive rate and a
reasonable false negative rate. The main limitation of this
setup is that the SVMs are trained to specific image sizes and each set
of SVMs require a data set of images of the same size for feature extraction.
\\
It may be possible to create a number of different sets of SVMs for
common image sizes in the data set. Most images are 358x268 pixels in
size but a few of them are slightly bigger or smaller. By training SVMs
on a set of clean 358x268 images, it would be possible to test a large
percentage of the dataset for steganographic content. Similar sets of SVMs can be created for other
image sizes in the collection.
\section{Improving the Performance of Stegdetect}
\label{sec:futurestegdetect}
%stegdetect does not compile on gcc4 out of hte box
% made few changes to make it to
% stegdetect links against an old version of the jpeg library
% need to update the jpeg lib version
% refactor code
% implement parallelisation
% Re write stegbreak as it often segfaults


\section{Realtime Image Steganography Detection}
\label{sec:futurereal}
Network based intrusion detection systems like SNORT provide means to
create plugins for the realtime detection of suspicious data. A number
of plugins exist for SNORT that provide malware/spyware detection at the
network level. A similar plugin using stegdetect for detection of
steganography can be created which would allow SNORT to flag and
store a copy of images suspected to contain steganographic content.
These images could later be analysed offline using a powerful system. 
